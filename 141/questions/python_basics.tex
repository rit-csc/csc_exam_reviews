Python basics
\begin{enumerate}

\item The programming language Python (circle the best answer):\\
	\begin{enumerate}
		\item is primarily a scripting language...
								\hspace{42mm}\textbf{True / False}\\
		\item supports the Object-Oriented paradigm...
								\hspace{30mm}\textbf{True / False}\\
		\item features static typing...
								\hspace{61mm}\textbf{True / False}\\
	\end{enumerate}
\begin{answer}
T, T, F
\end{answer}


\item Write an iterative function to print out all numbers from 1 to \emph{n} and then return the sum.

\begin{answer}
\begin{lstlisting}
def foo(n):
	total = 0
	for num in range(1,n+1):
		print(num)
		total+=num
	return total
\end{lstlisting}
\end{answer}

\item Re-write the function above using recursion. Is your answer tail recursive?  How do you know?

\begin{answer}
The answer described below \emph{is} tail recursive because no additional information must be stored on the stack for each recursive call.
.
\begin{lstlisting}
def foo_rec(limit, num=1, total=0):
	if num > limit:
		return total
	print(num)
	total+=num
	num+=1
	return foo_rec(limit, num, total)
\end{lstlisting}

However, if the line with the return statement included adding 1 to the return value (i.e., \texttt{return $1 +$ rec\_func$()$}), all of those \texttt{+1}s for the recursive calls would have to be stored on the stack, which would make the function \emph{not} tail recursive.
\end{answer}

\item How would you test the two functions you've written above? Explain test cases you would want to analyze.

\begin{answer}
\begin{itemize}
\item Extraneous cases: n=None, n is not an integer
\item Base cases: $n=0$, $n=1$
\item General case: $n > 1$
\end{itemize}
Since the functions described above are fruitful (have return values), we can write test cases if we know what
value we should expect before we run the function. This can be done with hard-coded values or using a separate algorithm/formula
to assert the validity of our function's return values.\\
\begin{lstlisting}
def test_foo_functions(n):
	# below suggests the formula for the sum 1+2+...+(n-1)+n
	expected = ( n * ( n + 1 ) ) // 2
	foo_iter_result = foo(n)
	if foo_iter_result != expected:
		print("iterative foo function produced an incorrect result!")
	foo_rec_result = foo_rec(n)
	if foo_rec_result != expected:
		print("recursive foo function produced an incorrect result!")
\end{lstlisting}
\end{answer}

\end{enumerate}




