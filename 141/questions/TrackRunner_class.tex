For the sake of this question, you find yourself to be the head programmer under Kim Jong Un's glorious reign.
It also just so happens that a nation-wide track meet is being held today. Thus, the glorious leader has demanded that
you write a program to keep track of information relating to all track runners present at the event.

\begin{enumerate}
\item Write a struct called \texttt{TrackRunner} to keep track of each competing runner.
You will need to store each runner's \texttt{name} (a string), \texttt{age} (an int), and \texttt{fastestTime} (an int).
\begin{answer}
\begin{lstlisting}[numbers=none]
from dataclasses import dataclass
@dataclass
class TrackRunner:
    '''Class to represent a TrackRunner'''
    name: str
    age: int
    fastestTime: int
\end{lstlisting}
\end{answer}

\item Now write a function to make an individual \texttt{TrackRunner} struct.
\begin{answer}
\begin{lstlisting}[numbers=none]
def makeRunner(name,age,fastestTime):
    return TrackRunner(name, age, fastestTime)
\end{lstlisting}
\end{answer}

\item The glorious leader has decided that, on this day, no runner named "Joe" may win
gold. Given a list of \texttt{TrackRunner} structs, write the function \texttt{aWinnerIsYou(runners)} that returns the runner in the list \texttt{runners} with the fastest time whose name is not ``Joe''.
Use this function to find the runner's name, age, and fastest time and print those results.
\begin{lstlisting}[numbers=none]
def aWinnerIsYou(runners):
\end{lstlisting}
\begin{answer}
\begin{lstlisting}[numbers=none]
    best = None
    for i in range(len(runners)):
        curr = runners[i]
        if curr.name != 'Joe':
            if best == None:
                best = curr
            elif best.fastestTime > curr.fastestTime:
                best = curr
    return best

winner = aWinnerIsYou(runnersLst)
print(winner.name, winner.age, winner.fastestTime)
\end{lstlisting}
\end{answer}
\end{enumerate}
