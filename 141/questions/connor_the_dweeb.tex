Connor is a big dweeb and loves keeping a count of things.
You will be writing a tail-recursive function to satisfy his desires.\\
\textit{Assume you are given the functions strHead() and strTail(), which run in constant time.\\
strHead() returns the first character in the string and strTail() returns the rest of the string.}
\begin{enumerate}

\item
\label{tailrecursive}
Write a tail recursive function \texttt{coRec}, which takes a string and a character and returns the number of times the character appears in the string. \\
For example, \texttt{coRec("Eric is enjoying the weather.", "i")} should return \texttt{3}. \\
Do not use the \texttt{len()} function.
\begin{answer}
\begin{lstlisting}
def coRec(string,char,counter=0):
    if(string==""):
        return counter
    if(strHead(string)==char):
        return coRec(strTail(string),char,counter+1)
    else:
        return coRec(strTail(string),char,counter)
\end{lstlisting}
\end{answer}

\item
What is the complexity of the functions you wrote for \ref{tailrecursive}?
\begin{answer}

It runs in O(N) time. \\
\end{answer}
\end{enumerate}
