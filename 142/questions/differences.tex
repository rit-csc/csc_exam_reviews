Explain the differences between:
\begin{enumerate}
	\item {\tt class} vs {\tt object} \\
	\begin{answer}
	A class defines methods and fields---it can be viewed as a template. An object is an instance of a class. Think of classes as molds and objects as individual things created by those molds.
	\end{answer}	
	
	\item constant vs non-constant field (variable). Declare a constant. \\
	\begin{answer}
	A constant field cannot be changed during run time (it may be set {\em once}).
		\begin{lstlisting}[numbers=none]
public final int NUM_PEOPLE_WHO_LIKE_JAVA = 1;
		\end{lstlisting}
	Note that marking mutable types {\tt final} only prevents them from being reassigned; they can still call methods that mutate themselves. For example, you can declare a {\tt final ArrayList}, but this does not prevent you from modifying the contents of it, such as through the use of the {\tt add(T toInsert)} method.
	\end{answer}
	
	\item final vs non-final method. \\
	\begin{answer}
	A final method can't be overridden by a subclass.
	\end{answer}

	\item class vs instance variable. Declare variables of both types.\\
	\begin{answer}
	Class (static) variables are scoped to the class, not to instances of that class. All instances of a class access the same static variable. However, instance (non-static) variables are unique to each instance.
		\begin{lstlisting}[numbers=none]
class Types {
	public static int im_a_class_var = 1;
	public int im_an_instance_var = 2;
}
		\end{lstlisting}
	\end{answer}
	
\end{enumerate}
