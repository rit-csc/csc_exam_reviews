Describe each of the following layout managers:
\begin{enumerate}

\item \texttt{FlowPane}
\begin{answer}
The \texttt{FlowPane} class puts components in a row, sized at their preferred size. If the horizontal space in the container is too small to put all the components in one row, the \texttt{FlowPane} class uses multiple rows. If the container is wider than necessary for a row of components, elements in the row will be flushed left by default. \end{answer}

\item \texttt{BorderPane}
\begin{answer}
\texttt{BorderPanes} allow you to specify the positional location to place new elements into the pane, including top, center, bottom, left, and right. You can specify where to put elements by \texttt{paneName.set[Position](Node value)}. Elements that are placed in the scene will retain their size if there is enough space.\end{answer}

\item \texttt{GridPane}
\begin{answer}
A \texttt{GridPane} places components in a grid of cells. The largest component in any given row or column dictates the size of that row or column, meaning if all your components are the same size, all the grid cells in the pane will be the same size.\end{answer} 

\item \texttt{HBox and VBox}
\begin{answer}
An \texttt{HBox} places its components horizonally left-to-right.
A \texttt{VBox} is like an \texttt{HBox} except that it adds components vertically top-to-bottom.
\end{answer}

\end{enumerate}

\vspace{24pt}
