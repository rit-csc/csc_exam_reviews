Explain the relationship between machine language, assembly language, and high level languages.\\
\begin{answer}
\small{\textbf{Machine language} is a set of instructions executed as-is by the computer's CPU. This is considered the lowest level representation of a computer program (other than binary) and, while it is possible to program directly in machine code, it is highly tedious and error prone, making higher level languages favorable. Writing machine code is typically only done when troubleshooting a system or when implementing extreme optimization.

\textbf{Assembly language} is the next step up from machine language and usually has a near 1:1 mapping from assembly code to the architecture's machine code instructions. \textit{Assembly languages are specific to a computer's architecture} -- this is because different architectures have different sets of hardware-supported commands. Programming in assembly language (and lower) is commonplace in embedded systems work.


Finally, \textbf{high level languages} are generally designed to be portable across many different architectures. What separates the high level languages from their lower level counterparts is that they require compilation, translation from high-level code into a form that the machine can understand (which varies by architecture.) High level languages are designed to be human-readable and abstract away some of the low-level details of programming. Some examples of high level languages are C/C++, Python, and Java.}
\end{answer}
